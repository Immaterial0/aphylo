\usepackage{booktabs}
\usepackage{amsmath, amssymb}

% Mathematical functions
\newcommand{\isone}[1]{{\boldsymbol{1}\left( #1 \right)}}
\renewcommand{\Pr}[1]{{\mathbb{P}\left(#1\right) }}
\newcommand{\f}[1]{{f\left(#1\right) }}
\newcommand{\Prcond}[2]{{\mathbb{P}\left(#1\vphantom{#2}\;\right|\left.\vphantom{#1}#2\right)}}
\newcommand{\fcond}[2]{{f\left(#1|#2\right) }}
\newcommand{\Expected}[1]{{\mathbb{E}\left\{#1\right\}}}

\newcommand{\Likelihood}[2]{\text{L}\left(#1 \left|\vphantom{#1}#2\right.\right)}

% Mathematical Annotation -------------------------------
% Modify this so that it matches the P01 convention overall

% Tree
\newcommand{\phylo}{\Lambda{}} % The actual tree
\newcommand{\aphylo}{D{}}      % The annotated phylogenetic tree
\newcommand{\aphyloObs}{\tilde \aphylo{}} % The observed annotated phylogenetic tree
\newcommand{\parent}[1]{\mathbf{p}\left(#1\right)}
\newcommand{\offspring}[1]{\mathbf{O}\left(#1\right)}
\newcommand{\nodes}{\mathcal{N}{}}
\newcommand{\edges}{\mathcal{E}{}}

% Annotations
\newcommand{\Ann}{\mathbf{X}{}} % Matrix of "real" annotations
\newcommand{\ann}[1]{x_{#1}{}} % single element of "real" annotations
\newcommand{\constraints}{\mathcal{C}{}} % Taxon constraints

% Obs Annotations
\newcommand{\AnnObs}{\mathbf{Z}{}}%{Z{}} \mathbf{X}^{obs}{}
\newcommand{\annObs}[1]{z_{#1}{}}%{z{}}  x_{#1}^{obs

% Pred. Annotations
\newcommand{\AnnPred}{\hat X{}}
\newcommand{\annPred}[1]{\hat x_{#1}}

% Leaf nodes
\newcommand{\Leaf}{L{}}

% Shortest path
\newcommand{\Geodesic}{\text{T}{}}
\newcommand{\geodesic}{\tau{}}

\newcommand{\Params}{\Omega{}}
\newcommand{\params}{\omega{}}

% Parameters
\newcommand{\gain}{\mu_{01}{}}
\newcommand{\loss}{\mu_{10}{}}
\newcommand{\misszero}{\psi_{01}{}}
\newcommand{\missone}{\psi_{10}{}}
\newcommand{\proot}{\pi}

% tricks for two column
\def\begincols{\begin{columns}[T]}
\def\begincol{\begin{column}[T]}
\def\endcol{\end{column}}
\def\endcols{\end{columns}}

\usepackage{tabularx}

\usepackage{tikz}

\newcommand{\includetikz}[2]{
\begin{figure}
\scalebox{#2}{
\input{#1}
}
\end{figure}
}